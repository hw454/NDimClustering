% Options for packages loaded elsewhere
\PassOptionsToPackage{unicode}{hyperref}
\PassOptionsToPackage{hyphens}{url}
%
\documentclass[
]{article}
\usepackage{amsmath,amssymb}
\usepackage{iftex}
\ifPDFTeX
  \usepackage[T1]{fontenc}
  \usepackage[utf8]{inputenc}
  \usepackage{textcomp} % provide euro and other symbols
\else % if luatex or xetex
  \usepackage{unicode-math} % this also loads fontspec
  \defaultfontfeatures{Scale=MatchLowercase}
  \defaultfontfeatures[\rmfamily]{Ligatures=TeX,Scale=1}
\fi
\usepackage{lmodern}
\ifPDFTeX\else
  % xetex/luatex font selection
\fi
% Use upquote if available, for straight quotes in verbatim environments
\IfFileExists{upquote.sty}{\usepackage{upquote}}{}
\IfFileExists{microtype.sty}{% use microtype if available
  \usepackage[]{microtype}
  \UseMicrotypeSet[protrusion]{basicmath} % disable protrusion for tt fonts
}{}
\makeatletter
\@ifundefined{KOMAClassName}{% if non-KOMA class
  \IfFileExists{parskip.sty}{%
    \usepackage{parskip}
  }{% else
    \setlength{\parindent}{0pt}
    \setlength{\parskip}{6pt plus 2pt minus 1pt}}
}{% if KOMA class
  \KOMAoptions{parskip=half}}
\makeatother
\usepackage{xcolor}
\usepackage[margin=1in]{geometry}
\usepackage{color}
\usepackage{fancyvrb}
\newcommand{\VerbBar}{|}
\newcommand{\VERB}{\Verb[commandchars=\\\{\}]}
\DefineVerbatimEnvironment{Highlighting}{Verbatim}{commandchars=\\\{\}}
% Add ',fontsize=\small' for more characters per line
\usepackage{framed}
\definecolor{shadecolor}{RGB}{248,248,248}
\newenvironment{Shaded}{\begin{snugshade}}{\end{snugshade}}
\newcommand{\AlertTok}[1]{\textcolor[rgb]{0.94,0.16,0.16}{#1}}
\newcommand{\AnnotationTok}[1]{\textcolor[rgb]{0.56,0.35,0.01}{\textbf{\textit{#1}}}}
\newcommand{\AttributeTok}[1]{\textcolor[rgb]{0.13,0.29,0.53}{#1}}
\newcommand{\BaseNTok}[1]{\textcolor[rgb]{0.00,0.00,0.81}{#1}}
\newcommand{\BuiltInTok}[1]{#1}
\newcommand{\CharTok}[1]{\textcolor[rgb]{0.31,0.60,0.02}{#1}}
\newcommand{\CommentTok}[1]{\textcolor[rgb]{0.56,0.35,0.01}{\textit{#1}}}
\newcommand{\CommentVarTok}[1]{\textcolor[rgb]{0.56,0.35,0.01}{\textbf{\textit{#1}}}}
\newcommand{\ConstantTok}[1]{\textcolor[rgb]{0.56,0.35,0.01}{#1}}
\newcommand{\ControlFlowTok}[1]{\textcolor[rgb]{0.13,0.29,0.53}{\textbf{#1}}}
\newcommand{\DataTypeTok}[1]{\textcolor[rgb]{0.13,0.29,0.53}{#1}}
\newcommand{\DecValTok}[1]{\textcolor[rgb]{0.00,0.00,0.81}{#1}}
\newcommand{\DocumentationTok}[1]{\textcolor[rgb]{0.56,0.35,0.01}{\textbf{\textit{#1}}}}
\newcommand{\ErrorTok}[1]{\textcolor[rgb]{0.64,0.00,0.00}{\textbf{#1}}}
\newcommand{\ExtensionTok}[1]{#1}
\newcommand{\FloatTok}[1]{\textcolor[rgb]{0.00,0.00,0.81}{#1}}
\newcommand{\FunctionTok}[1]{\textcolor[rgb]{0.13,0.29,0.53}{\textbf{#1}}}
\newcommand{\ImportTok}[1]{#1}
\newcommand{\InformationTok}[1]{\textcolor[rgb]{0.56,0.35,0.01}{\textbf{\textit{#1}}}}
\newcommand{\KeywordTok}[1]{\textcolor[rgb]{0.13,0.29,0.53}{\textbf{#1}}}
\newcommand{\NormalTok}[1]{#1}
\newcommand{\OperatorTok}[1]{\textcolor[rgb]{0.81,0.36,0.00}{\textbf{#1}}}
\newcommand{\OtherTok}[1]{\textcolor[rgb]{0.56,0.35,0.01}{#1}}
\newcommand{\PreprocessorTok}[1]{\textcolor[rgb]{0.56,0.35,0.01}{\textit{#1}}}
\newcommand{\RegionMarkerTok}[1]{#1}
\newcommand{\SpecialCharTok}[1]{\textcolor[rgb]{0.81,0.36,0.00}{\textbf{#1}}}
\newcommand{\SpecialStringTok}[1]{\textcolor[rgb]{0.31,0.60,0.02}{#1}}
\newcommand{\StringTok}[1]{\textcolor[rgb]{0.31,0.60,0.02}{#1}}
\newcommand{\VariableTok}[1]{\textcolor[rgb]{0.00,0.00,0.00}{#1}}
\newcommand{\VerbatimStringTok}[1]{\textcolor[rgb]{0.31,0.60,0.02}{#1}}
\newcommand{\WarningTok}[1]{\textcolor[rgb]{0.56,0.35,0.01}{\textbf{\textit{#1}}}}
\usepackage{graphicx}
\makeatletter
\def\maxwidth{\ifdim\Gin@nat@width>\linewidth\linewidth\else\Gin@nat@width\fi}
\def\maxheight{\ifdim\Gin@nat@height>\textheight\textheight\else\Gin@nat@height\fi}
\makeatother
% Scale images if necessary, so that they will not overflow the page
% margins by default, and it is still possible to overwrite the defaults
% using explicit options in \includegraphics[width, height, ...]{}
\setkeys{Gin}{width=\maxwidth,height=\maxheight,keepaspectratio}
% Set default figure placement to htbp
\makeatletter
\def\fps@figure{htbp}
\makeatother
\setlength{\emergencystretch}{3em} % prevent overfull lines
\providecommand{\tightlist}{%
  \setlength{\itemsep}{0pt}\setlength{\parskip}{0pt}}
\setcounter{secnumdepth}{-\maxdimen} % remove section numbering
\ifLuaTeX
  \usepackage{selnolig}  % disable illegal ligatures
\fi
\IfFileExists{bookmark.sty}{\usepackage{bookmark}}{\usepackage{hyperref}}
\IfFileExists{xurl.sty}{\usepackage{xurl}}{} % add URL line breaks if available
\urlstyle{same}
\hypersetup{
  pdftitle={Testing Clustering methods.},
  pdfauthor={Hayley Wragg},
  hidelinks,
  pdfcreator={LaTeX via pandoc}}

\title{Testing Clustering methods.}
\author{Hayley Wragg}
\date{21st Nov 2023}

\begin{document}
\maketitle

\hypertarget{install-package}{%
\section{Install package}\label{install-package}}

Load the ``ndimclusteringR'' package

\begin{Shaded}
\begin{Highlighting}[]
\NormalTok{devtools}\SpecialCharTok{::}\FunctionTok{install}\NormalTok{(}\StringTok{"../ndimclusteringR/R"}\NormalTok{)}
\end{Highlighting}
\end{Shaded}

\begin{verbatim}
## scales (1.2.1 -> 1.3.0) [CRAN]
\end{verbatim}

\begin{verbatim}
## Installing 1 packages: scales
\end{verbatim}

\begin{verbatim}
## Installing package into '/home/hayley/R/x86_64-pc-linux-gnu-library/4.3'
## (as 'lib' is unspecified)
\end{verbatim}

\begin{verbatim}
## -- R CMD build -----------------------------------------------------------------
##      checking for file ‘/home/hayley/Code/ICEP/NDimClustering/ndimclusteringR/DESCRIPTION’ ...  v  checking for file ‘/home/hayley/Code/ICEP/NDimClustering/ndimclusteringR/DESCRIPTION’ (339ms)
##   -  preparing ‘ndimclusteringR’:
##    checking DESCRIPTION meta-information ...  v  checking DESCRIPTION meta-information
##   -  checking for LF line-endings in source and make files and shell scripts
##   -  checking for empty or unneeded directories
##   -  building ‘ndimclusteringR_0.0.1.tar.gz’
##      
## Running /usr/lib/R/bin/R CMD INSTALL \
##   /tmp/RtmpCiHeeC/ndimclusteringR_0.0.1.tar.gz --install-tests 
## * installing to library ‘/home/hayley/R/x86_64-pc-linux-gnu-library/4.3’
## * installing *source* package ‘ndimclusteringR’ ...
## ** using staged installation
## ** R
## ** data
## *** moving datasets to lazyload DB
## ** byte-compile and prepare package for lazy loading
## ** help
## *** installing help indices
## ** building package indices
## ** testing if installed package can be loaded from temporary location
## ** testing if installed package can be loaded from final location
## ** testing if installed package keeps a record of temporary installation path
## * DONE (ndimclusteringR)
\end{verbatim}

\begin{Shaded}
\begin{Highlighting}[]
\FunctionTok{library}\NormalTok{(}\StringTok{"ndimclusteringR"}\NormalTok{)}
\end{Highlighting}
\end{Shaded}

\hypertarget{source-test}{%
\section{Source test}\label{source-test}}

Load the functions for testing

\begin{Shaded}
\begin{Highlighting}[]
\FunctionTok{source}\NormalTok{(}\StringTok{"test\_clust\_kmeans.R"}\NormalTok{)}
\end{Highlighting}
\end{Shaded}

\hypertarget{step-by-step}{%
\section{Step by step}\label{step-by-step}}

If all the steps are included then they are performed in the following
order.

1. Crop data to complete cases and lowest se.

2. Convert data to angles.

3. Perform PCA on angle data.

4. Cluster on the PCA

\begin{itemize}
\tightlist
\item
  Set cluster centres
\item
  Converge cluster centres
\item
  if min then rerun for different number of clusters and choose set with
  min-aic.
\end{itemize}

\hypertarget{test-1}{%
\subsubsection{Test 1}\label{test-1}}

Run the tests with:

\begin{itemize}
\tightlist
\item
  No PCA
\item
  basic k-means clustering. k is set to the number of pathways.
\item
  no angles calculated before clustering.
\item
  number of pathways 2.
\end{itemize}

\begin{Shaded}
\begin{Highlighting}[]
\NormalTok{pc\_type }\OtherTok{\textless{}{-}} \StringTok{"No\_pca"}
\NormalTok{np }\OtherTok{\textless{}{-}} \DecValTok{1}
\NormalTok{d }\OtherTok{\textless{}{-}} \DecValTok{30}
\NormalTok{clust\_typ }\OtherTok{\textless{}{-}} \StringTok{"basic"}
\NormalTok{space\_typ }\OtherTok{\textless{}{-}} \StringTok{"regular"}
\FunctionTok{test\_clust\_kmeans\_function}\NormalTok{(d,}
                           \AttributeTok{num\_path =}\NormalTok{ np,}
                           \AttributeTok{pc\_type =}\NormalTok{ pc\_type,}
                           \AttributeTok{clust\_typ =}\NormalTok{ clust\_typ,}
                           \AttributeTok{space\_typ =}\NormalTok{ space\_typ)}
\end{Highlighting}
\end{Shaded}

\begin{verbatim}
## [1] "Test No_pca with 2 pathways"
##   bp_on clust_prob_on         clust_typ ndim_typ how_cents pc_type num_paths
## 1  TRUE          TRUE test_basicregular test_all      rand  No_pca         1
##           res_dir
## 1 PC_TestResults/
## [1] "Clusters converged 5"
\end{verbatim}

\includegraphics{test_clust_kmeans_files/figure-latex/unnamed-chunk-3-1.pdf}
\includegraphics{test_clust_kmeans_files/figure-latex/unnamed-chunk-3-2.pdf}
\includegraphics{test_clust_kmeans_files/figure-latex/unnamed-chunk-3-3.pdf}

\hypertarget{test-2}{%
\subsubsection{Test 2}\label{test-2}}

Run the tests with:

\begin{itemize}
\tightlist
\item
  No PCA
\item
  basic k-means clustering. k is set to the number of pathways.
\item
  no angles calculated before clustering.
\item
  number of pathways 4.
\end{itemize}

\begin{Shaded}
\begin{Highlighting}[]
\NormalTok{pc\_type }\OtherTok{\textless{}{-}} \StringTok{"No\_pca"}
\NormalTok{np }\OtherTok{\textless{}{-}} \DecValTok{3}
\NormalTok{d }\OtherTok{\textless{}{-}} \DecValTok{30}
\NormalTok{clust\_typ }\OtherTok{\textless{}{-}} \StringTok{"basic"}
\NormalTok{space\_typ }\OtherTok{\textless{}{-}} \StringTok{"regular"}
\FunctionTok{test\_clust\_kmeans\_function}\NormalTok{(d,}
                           \AttributeTok{num\_path =}\NormalTok{ np,}
                           \AttributeTok{pc\_type =}\NormalTok{ pc\_type,}
                           \AttributeTok{clust\_typ =}\NormalTok{ clust\_typ,}
                           \AttributeTok{space\_typ =}\NormalTok{ space\_typ)}
\end{Highlighting}
\end{Shaded}

\begin{verbatim}
## [1] "Test No_pca with 4 pathways"
##   bp_on clust_prob_on         clust_typ ndim_typ how_cents pc_type num_paths
## 1  TRUE          TRUE test_basicregular test_all      rand  No_pca         3
##           res_dir
## 1 PC_TestResults/
## [1] "Clusters converged 3"
\end{verbatim}

\includegraphics{test_clust_kmeans_files/figure-latex/unnamed-chunk-4-1.pdf}
\includegraphics{test_clust_kmeans_files/figure-latex/unnamed-chunk-4-2.pdf}
\includegraphics{test_clust_kmeans_files/figure-latex/unnamed-chunk-4-3.pdf}

\hypertarget{test-3}{%
\subsubsection{Test 3}\label{test-3}}

Run the tests with:

\begin{itemize}
\tightlist
\item
  No PCA
\item
  basic k-means clustering. k is set to the number of pathways.
\item
  angles calculated before pca and before clustering.
\item
  number of pathways 2.
\end{itemize}

\begin{Shaded}
\begin{Highlighting}[]
\NormalTok{pc\_type }\OtherTok{\textless{}{-}} \StringTok{"No\_pca"}
\NormalTok{np }\OtherTok{\textless{}{-}} \DecValTok{1}
\NormalTok{d }\OtherTok{\textless{}{-}} \DecValTok{30}
\NormalTok{clust\_typ }\OtherTok{\textless{}{-}} \StringTok{"basic"}
\NormalTok{space\_typ }\OtherTok{\textless{}{-}} \StringTok{"angle"}
\FunctionTok{test\_clust\_kmeans\_function}\NormalTok{(d,}
                           \AttributeTok{num\_path =}\NormalTok{ np,}
                           \AttributeTok{pc\_type =}\NormalTok{ pc\_type,}
                           \AttributeTok{clust\_typ =}\NormalTok{ clust\_typ,}
                           \AttributeTok{space\_typ =}\NormalTok{ space\_typ)}
\end{Highlighting}
\end{Shaded}

\begin{verbatim}
## [1] "Test No_pca with 2 pathways"
##   bp_on clust_prob_on       clust_typ ndim_typ how_cents pc_type num_paths
## 1  TRUE          TRUE test_basicangle test_all      rand  No_pca         1
##           res_dir
## 1 PC_TestResults/
## [1] "Clusters converged 3"
\end{verbatim}

\includegraphics{test_clust_kmeans_files/figure-latex/unnamed-chunk-5-1.pdf}
\includegraphics{test_clust_kmeans_files/figure-latex/unnamed-chunk-5-2.pdf}
\includegraphics{test_clust_kmeans_files/figure-latex/unnamed-chunk-5-3.pdf}
\includegraphics{test_clust_kmeans_files/figure-latex/unnamed-chunk-5-4.pdf}

\hypertarget{test-4}{%
\subsubsection{Test 4}\label{test-4}}

Run the tests with:

\begin{itemize}
\tightlist
\item
  No PCA
\item
  basic k-means clustering. k is set to the number of pathways.
\item
  angles calculated before clustering.
\item
  number of pathways 4.
\end{itemize}

\begin{Shaded}
\begin{Highlighting}[]
\NormalTok{pc\_type }\OtherTok{\textless{}{-}} \StringTok{"No\_pca"}
\NormalTok{np }\OtherTok{\textless{}{-}} \DecValTok{3}
\NormalTok{d }\OtherTok{\textless{}{-}} \DecValTok{30}
\NormalTok{clust\_typ }\OtherTok{\textless{}{-}} \StringTok{"basic"}
\NormalTok{space\_typ }\OtherTok{\textless{}{-}} \StringTok{"angle"}
\FunctionTok{test\_clust\_kmeans\_function}\NormalTok{(d,}
                           \AttributeTok{num\_path =}\NormalTok{ np,}
                           \AttributeTok{pc\_type =}\NormalTok{ pc\_type,}
                           \AttributeTok{clust\_typ =}\NormalTok{ clust\_typ,}
                           \AttributeTok{space\_typ =}\NormalTok{ space\_typ)}
\end{Highlighting}
\end{Shaded}

\begin{verbatim}
## [1] "Test No_pca with 4 pathways"
##   bp_on clust_prob_on       clust_typ ndim_typ how_cents pc_type num_paths
## 1  TRUE          TRUE test_basicangle test_all      rand  No_pca         3
##           res_dir
## 1 PC_TestResults/
## [1] "Clusters converged 6"
\end{verbatim}

\includegraphics{test_clust_kmeans_files/figure-latex/unnamed-chunk-6-1.pdf}
\includegraphics{test_clust_kmeans_files/figure-latex/unnamed-chunk-6-2.pdf}
\includegraphics{test_clust_kmeans_files/figure-latex/unnamed-chunk-6-3.pdf}
\includegraphics{test_clust_kmeans_files/figure-latex/unnamed-chunk-6-4.pdf}

\hypertarget{test-5}{%
\subsubsection{Test 5}\label{test-5}}

Run the tests with:

\begin{itemize}
\tightlist
\item
  No PCA
\item
  min-aic k-means clustering. Maximum k is set to the number of
  pathways.
\item
  no angles calculated before clustering.
\item
  number of pathways 2.
\end{itemize}

\begin{Shaded}
\begin{Highlighting}[]
\NormalTok{pc\_type }\OtherTok{\textless{}{-}} \StringTok{"No\_pca"}
\NormalTok{np }\OtherTok{\textless{}{-}} \DecValTok{1}
\NormalTok{d }\OtherTok{\textless{}{-}} \DecValTok{30}
\NormalTok{clust\_typ }\OtherTok{\textless{}{-}} \StringTok{"min"}
\NormalTok{space\_typ }\OtherTok{\textless{}{-}} \StringTok{"regular"}
\FunctionTok{test\_clust\_kmeans\_function}\NormalTok{(d,}
                           \AttributeTok{num\_path =}\NormalTok{ np,}
                           \AttributeTok{pc\_type =}\NormalTok{ pc\_type,}
                           \AttributeTok{clust\_typ =}\NormalTok{ clust\_typ,}
                           \AttributeTok{space\_typ =}\NormalTok{ space\_typ)}
\end{Highlighting}
\end{Shaded}

\begin{verbatim}
## [1] "Test No_pca with 2 pathways"
##   bp_on clust_prob_on       clust_typ ndim_typ how_cents pc_type num_paths
## 1  TRUE          TRUE test_minregular test_all      rand  No_pca         1
##           res_dir
## 1 PC_TestResults/
## [1] "Clusters converged 2"
## [1] "Clusters converged 5"
## [1] "Clusters converged 3"
\end{verbatim}

\includegraphics{test_clust_kmeans_files/figure-latex/unnamed-chunk-7-1.pdf}
\includegraphics{test_clust_kmeans_files/figure-latex/unnamed-chunk-7-2.pdf}
\includegraphics{test_clust_kmeans_files/figure-latex/unnamed-chunk-7-3.pdf}

\hypertarget{test-6}{%
\subsubsection{Test 6}\label{test-6}}

Run the tests with:

\begin{itemize}
\tightlist
\item
  No PCA
\item
  min-aic k-means clustering. Maximum k is set to the number of
  pathways.
\item
  no angles calculated before clustering.
\item
  number of pathways 4.
\end{itemize}

\begin{Shaded}
\begin{Highlighting}[]
\NormalTok{pc\_type }\OtherTok{\textless{}{-}} \StringTok{"No\_pca"}
\NormalTok{np }\OtherTok{\textless{}{-}} \DecValTok{3}
\NormalTok{d }\OtherTok{\textless{}{-}} \DecValTok{30}
\NormalTok{clust\_typ }\OtherTok{\textless{}{-}} \StringTok{"min"}
\NormalTok{space\_typ }\OtherTok{\textless{}{-}} \StringTok{"regular"}
\FunctionTok{test\_clust\_kmeans\_function}\NormalTok{(d,}
                           \AttributeTok{num\_path =}\NormalTok{ np,}
                           \AttributeTok{pc\_type =}\NormalTok{ pc\_type,}
                           \AttributeTok{clust\_typ =}\NormalTok{ clust\_typ,}
                           \AttributeTok{space\_typ =}\NormalTok{ space\_typ)}
\end{Highlighting}
\end{Shaded}

\begin{verbatim}
## [1] "Test No_pca with 4 pathways"
##   bp_on clust_prob_on       clust_typ ndim_typ how_cents pc_type num_paths
## 1  TRUE          TRUE test_minregular test_all      rand  No_pca         3
##           res_dir
## 1 PC_TestResults/
## [1] "Clusters converged 2"
## [1] "Clusters converged 4"
## [1] "Clusters converged 3"
## [1] "Clusters converged 6"
## [1] "Clusters converged 5"
\end{verbatim}

\includegraphics{test_clust_kmeans_files/figure-latex/unnamed-chunk-8-1.pdf}
\includegraphics{test_clust_kmeans_files/figure-latex/unnamed-chunk-8-2.pdf}
\includegraphics{test_clust_kmeans_files/figure-latex/unnamed-chunk-8-3.pdf}

\hypertarget{test-7}{%
\subsubsection{Test 7}\label{test-7}}

Run the tests with:

\begin{itemize}
\tightlist
\item
  No PCA
\item
  min k-means clustering. Maximum k is set to the number of pathways.
\item
  angles calculated before clustering.
\item
  number of pathways 2.
\end{itemize}

\begin{Shaded}
\begin{Highlighting}[]
\NormalTok{pc\_type }\OtherTok{\textless{}{-}} \StringTok{"No\_pca"}
\NormalTok{np }\OtherTok{\textless{}{-}} \DecValTok{1}
\NormalTok{d }\OtherTok{\textless{}{-}} \DecValTok{30}
\NormalTok{clust\_typ }\OtherTok{\textless{}{-}} \StringTok{"min"}
\NormalTok{space\_typ }\OtherTok{\textless{}{-}} \StringTok{"angle"}
\FunctionTok{test\_clust\_kmeans\_function}\NormalTok{(d,}
                           \AttributeTok{num\_path =}\NormalTok{ np,}
                           \AttributeTok{pc\_type =}\NormalTok{ pc\_type,}
                           \AttributeTok{clust\_typ =}\NormalTok{ clust\_typ,}
                           \AttributeTok{space\_typ =}\NormalTok{ space\_typ)}
\end{Highlighting}
\end{Shaded}

\begin{verbatim}
## [1] "Test No_pca with 2 pathways"
##   bp_on clust_prob_on     clust_typ ndim_typ how_cents pc_type num_paths
## 1  TRUE          TRUE test_minangle test_all      rand  No_pca         1
##           res_dir
## 1 PC_TestResults/
## [1] "Clusters converged 2"
## [1] "Clusters converged 4"
## [1] "Clusters converged 3"
\end{verbatim}

\includegraphics{test_clust_kmeans_files/figure-latex/unnamed-chunk-9-1.pdf}
\includegraphics{test_clust_kmeans_files/figure-latex/unnamed-chunk-9-2.pdf}
\includegraphics{test_clust_kmeans_files/figure-latex/unnamed-chunk-9-3.pdf}
\includegraphics{test_clust_kmeans_files/figure-latex/unnamed-chunk-9-4.pdf}

\hypertarget{test-8}{%
\subsubsection{Test 8}\label{test-8}}

Run the tests with:

\begin{itemize}
\tightlist
\item
  No PCA
\item
  min-aic k-means clustering. Maximum k is set to the number of
  pathways.
\item
  angles calculated before clustering.
\item
  number of pathways 4.
\end{itemize}

\begin{Shaded}
\begin{Highlighting}[]
\NormalTok{pc\_type }\OtherTok{\textless{}{-}} \StringTok{"No\_pca"}
\NormalTok{np }\OtherTok{\textless{}{-}} \DecValTok{3}
\NormalTok{d }\OtherTok{\textless{}{-}} \DecValTok{30}
\NormalTok{clust\_typ }\OtherTok{\textless{}{-}} \StringTok{"min"}
\NormalTok{space\_typ }\OtherTok{\textless{}{-}} \StringTok{"angle"}
\FunctionTok{test\_clust\_kmeans\_function}\NormalTok{(d,}
                           \AttributeTok{num\_path =}\NormalTok{ np,}
                           \AttributeTok{pc\_type =}\NormalTok{ pc\_type,}
                           \AttributeTok{clust\_typ =}\NormalTok{ clust\_typ,}
                           \AttributeTok{space\_typ =}\NormalTok{ space\_typ)}
\end{Highlighting}
\end{Shaded}

\begin{verbatim}
## [1] "Test No_pca with 4 pathways"
##   bp_on clust_prob_on     clust_typ ndim_typ how_cents pc_type num_paths
## 1  TRUE          TRUE test_minangle test_all      rand  No_pca         3
##           res_dir
## 1 PC_TestResults/
## [1] "Clusters converged 2"
## [1] "Clusters converged 4"
## [1] "Clusters converged 3"
## [1] "Clusters converged 6"
## [1] "Clusters converged 5"
\end{verbatim}

\includegraphics{test_clust_kmeans_files/figure-latex/unnamed-chunk-10-1.pdf}
\includegraphics{test_clust_kmeans_files/figure-latex/unnamed-chunk-10-2.pdf}
\includegraphics{test_clust_kmeans_files/figure-latex/unnamed-chunk-10-3.pdf}
\includegraphics{test_clust_kmeans_files/figure-latex/unnamed-chunk-10-4.pdf}

\hypertarget{consider-the-location-of-the-centroids.}{%
\subsection{Consider the location of the
centroids.}\label{consider-the-location-of-the-centroids.}}

The previous tests used centroids assigned randomly within the ranges on
each axis. Instead we will initialise the centroids using randomly
selected points from out dataset.

\hypertarget{test-9}{%
\subsubsection{Test 9}\label{test-9}}

Run the tests with:

\begin{itemize}
\tightlist
\item
  No PCA
\item
  basic k-means clustering. k is set to the number of pathways.
\item
  no angles calculated before clustering.
\item
  number of pathways 4.
\item
  Centroids assigned using points
\end{itemize}

\begin{Shaded}
\begin{Highlighting}[]
\NormalTok{pc\_type }\OtherTok{\textless{}{-}} \StringTok{"No\_pca"}
\NormalTok{np }\OtherTok{\textless{}{-}} \DecValTok{3}
\NormalTok{d }\OtherTok{\textless{}{-}} \DecValTok{30}
\NormalTok{clust\_typ }\OtherTok{\textless{}{-}} \StringTok{"basic"}
\NormalTok{space\_typ }\OtherTok{\textless{}{-}} \StringTok{"regular"}
\NormalTok{how\_cents }\OtherTok{\textless{}{-}} \StringTok{"points"}
\FunctionTok{test\_clust\_kmeans\_function}\NormalTok{(d,}
                           \AttributeTok{num\_path =}\NormalTok{ np,}
                           \AttributeTok{pc\_type =}\NormalTok{ pc\_type,}
                           \AttributeTok{clust\_typ =}\NormalTok{ clust\_typ,}
                           \AttributeTok{space\_typ =}\NormalTok{ space\_typ,}
                           \AttributeTok{how\_cents =}\NormalTok{ how\_cents)}
\end{Highlighting}
\end{Shaded}

\begin{verbatim}
## [1] "Test No_pca with 4 pathways"
##   bp_on clust_prob_on         clust_typ ndim_typ how_cents pc_type num_paths
## 1  TRUE          TRUE test_basicregular test_all    points  No_pca         3
##           res_dir
## 1 PC_TestResults/
## [1] "Clusters converged 6"
\end{verbatim}

\includegraphics{test_clust_kmeans_files/figure-latex/unnamed-chunk-11-1.pdf}
\includegraphics{test_clust_kmeans_files/figure-latex/unnamed-chunk-11-2.pdf}
\includegraphics{test_clust_kmeans_files/figure-latex/unnamed-chunk-11-3.pdf}
\includegraphics{test_clust_kmeans_files/figure-latex/unnamed-chunk-11-4.pdf}

\hypertarget{test-10}{%
\subsubsection{Test 10}\label{test-10}}

Run the tests with: - No PCA - basic k-means clustering. k is set to the
number of pathways. - angles calculated before clustering. - number of
pathways 4. - Centroids assigned using points

\begin{Shaded}
\begin{Highlighting}[]
\NormalTok{pc\_type }\OtherTok{\textless{}{-}} \StringTok{"No\_pca"}
\NormalTok{np }\OtherTok{\textless{}{-}} \DecValTok{3}
\NormalTok{d }\OtherTok{\textless{}{-}} \DecValTok{30}
\NormalTok{clust\_typ }\OtherTok{\textless{}{-}} \StringTok{"basic"}
\NormalTok{space\_typ }\OtherTok{\textless{}{-}} \StringTok{"angle"}
\NormalTok{how\_cents }\OtherTok{\textless{}{-}} \StringTok{"points"}
\FunctionTok{test\_clust\_kmeans\_function}\NormalTok{(d,}
                           \AttributeTok{num\_path =}\NormalTok{ np,}
                           \AttributeTok{pc\_type =}\NormalTok{ pc\_type,}
                           \AttributeTok{clust\_typ =}\NormalTok{ clust\_typ,}
                           \AttributeTok{space\_typ =}\NormalTok{ space\_typ,}
                           \AttributeTok{how\_cents =}\NormalTok{ how\_cents)}
\end{Highlighting}
\end{Shaded}

\begin{verbatim}
## [1] "Test No_pca with 4 pathways"
##   bp_on clust_prob_on       clust_typ ndim_typ how_cents pc_type num_paths
## 1  TRUE          TRUE test_basicangle test_all    points  No_pca         3
##           res_dir
## 1 PC_TestResults/
## [1] "Clusters converged 2"
\end{verbatim}

\includegraphics{test_clust_kmeans_files/figure-latex/unnamed-chunk-12-1.pdf}
\includegraphics{test_clust_kmeans_files/figure-latex/unnamed-chunk-12-2.pdf}
\includegraphics{test_clust_kmeans_files/figure-latex/unnamed-chunk-12-3.pdf}
\includegraphics{test_clust_kmeans_files/figure-latex/unnamed-chunk-12-4.pdf}
\includegraphics{test_clust_kmeans_files/figure-latex/unnamed-chunk-12-5.pdf}

\hypertarget{test-11}{%
\subsubsection{Test 11}\label{test-11}}

Run the tests with:

\begin{itemize}
\tightlist
\item
  PCA with prcomp
\item
  basic k-means clustering. k is set to the number of pathways.
\item
  no angles calculated before clustering.
\item
  number of pathways 4.
\item
  Centroids assigned using points
\end{itemize}

\begin{Shaded}
\begin{Highlighting}[]
\NormalTok{pc\_type }\OtherTok{\textless{}{-}} \StringTok{"prcomp"}
\NormalTok{np }\OtherTok{\textless{}{-}} \DecValTok{3}
\NormalTok{d }\OtherTok{\textless{}{-}} \DecValTok{30}
\NormalTok{clust\_typ }\OtherTok{\textless{}{-}} \StringTok{"basic"}
\NormalTok{space\_typ }\OtherTok{\textless{}{-}} \StringTok{"regular"}
\NormalTok{how\_cents }\OtherTok{\textless{}{-}} \StringTok{"points"}
\FunctionTok{test\_clust\_kmeans\_function}\NormalTok{(d,}
                           \AttributeTok{num\_path =}\NormalTok{ np,}
                           \AttributeTok{pc\_type =}\NormalTok{ pc\_type,}
                           \AttributeTok{clust\_typ =}\NormalTok{ clust\_typ,}
                           \AttributeTok{space\_typ =}\NormalTok{ space\_typ,}
                           \AttributeTok{how\_cents =}\NormalTok{ how\_cents)}
\end{Highlighting}
\end{Shaded}

\begin{verbatim}
## [1] "Test prcomp with 4 pathways"
##   bp_on clust_prob_on         clust_typ ndim_typ how_cents pc_type num_paths
## 1  TRUE          TRUE test_basicregular test_all    points  prcomp         3
##           res_dir
## 1 PC_TestResults/
## [1] "Clusters converged 6"
\end{verbatim}

\includegraphics{test_clust_kmeans_files/figure-latex/unnamed-chunk-13-1.pdf}
\includegraphics{test_clust_kmeans_files/figure-latex/unnamed-chunk-13-2.pdf}
\includegraphics{test_clust_kmeans_files/figure-latex/unnamed-chunk-13-3.pdf}
\includegraphics{test_clust_kmeans_files/figure-latex/unnamed-chunk-13-4.pdf}

\hypertarget{test-12}{%
\subsubsection{Test 12}\label{test-12}}

Run the tests with:

\begin{itemize}
\tightlist
\item
  PCA with prcomp
\item
  basic k-means clustering. k is set to the number of pathways.
\item
  angles calculated before PCA and clustering.
\item
  number of pathways 4.
\item
  Centroids assigned using points
\end{itemize}

\begin{Shaded}
\begin{Highlighting}[]
\NormalTok{pc\_type }\OtherTok{\textless{}{-}} \StringTok{"prcomp"}
\NormalTok{np }\OtherTok{\textless{}{-}} \DecValTok{3}
\NormalTok{d }\OtherTok{\textless{}{-}} \DecValTok{30}
\NormalTok{clust\_typ }\OtherTok{\textless{}{-}} \StringTok{"basic"}
\NormalTok{space\_typ }\OtherTok{\textless{}{-}} \StringTok{"angle"}
\NormalTok{how\_cents }\OtherTok{\textless{}{-}} \StringTok{"points"}
\FunctionTok{test\_clust\_kmeans\_function}\NormalTok{(d,}
                           \AttributeTok{num\_path =}\NormalTok{ np,}
                           \AttributeTok{pc\_type =}\NormalTok{ pc\_type,}
                           \AttributeTok{clust\_typ =}\NormalTok{ clust\_typ,}
                           \AttributeTok{space\_typ =}\NormalTok{ space\_typ,}
                           \AttributeTok{how\_cents =}\NormalTok{ how\_cents)}
\end{Highlighting}
\end{Shaded}

\begin{verbatim}
## [1] "Test prcomp with 4 pathways"
##   bp_on clust_prob_on       clust_typ ndim_typ how_cents pc_type num_paths
## 1  TRUE          TRUE test_basicangle test_all    points  prcomp         3
##           res_dir
## 1 PC_TestResults/
## [1] "Clusters converged 5"
\end{verbatim}

\includegraphics{test_clust_kmeans_files/figure-latex/unnamed-chunk-14-1.pdf}
\includegraphics{test_clust_kmeans_files/figure-latex/unnamed-chunk-14-2.pdf}
\includegraphics{test_clust_kmeans_files/figure-latex/unnamed-chunk-14-3.pdf}
\includegraphics{test_clust_kmeans_files/figure-latex/unnamed-chunk-14-4.pdf}
\includegraphics{test_clust_kmeans_files/figure-latex/unnamed-chunk-14-5.pdf}

\hypertarget{test-13}{%
\subsubsection{Test 13}\label{test-13}}

Run the tests with:

\begin{itemize}
\tightlist
\item
  No PCA
\item
  min-aic k-means clustering. k is set to the number of pathways.
\item
  no angles calculated before clustering.
\item
  number of pathways 4.
\item
  Centroids assigned using points
\end{itemize}

\begin{Shaded}
\begin{Highlighting}[]
\NormalTok{pc\_type }\OtherTok{\textless{}{-}} \StringTok{"No\_pca"}
\NormalTok{np }\OtherTok{\textless{}{-}} \DecValTok{3}
\NormalTok{d }\OtherTok{\textless{}{-}} \DecValTok{30}
\NormalTok{clust\_typ }\OtherTok{\textless{}{-}} \StringTok{"min"}
\NormalTok{space\_typ }\OtherTok{\textless{}{-}} \StringTok{"regular"}
\NormalTok{how\_cents }\OtherTok{\textless{}{-}} \StringTok{"points"}
\FunctionTok{test\_clust\_kmeans\_function}\NormalTok{(d,}
                           \AttributeTok{num\_path =}\NormalTok{ np,}
                           \AttributeTok{pc\_type =}\NormalTok{ pc\_type,}
                           \AttributeTok{clust\_typ =}\NormalTok{ clust\_typ,}
                           \AttributeTok{space\_typ =}\NormalTok{ space\_typ,}
                           \AttributeTok{how\_cents =}\NormalTok{ how\_cents)}
\end{Highlighting}
\end{Shaded}

\begin{verbatim}
## [1] "Test No_pca with 4 pathways"
##   bp_on clust_prob_on       clust_typ ndim_typ how_cents pc_type num_paths
## 1  TRUE          TRUE test_minregular test_all    points  No_pca         3
##           res_dir
## 1 PC_TestResults/
## [1] "Clusters converged 2"
## [1] "Clusters converged 2"
## [1] "Clusters converged 4"
## [1] "Clusters converged 3"
## [1] "Clusters converged 4"
\end{verbatim}

\includegraphics{test_clust_kmeans_files/figure-latex/unnamed-chunk-15-1.pdf}
\includegraphics{test_clust_kmeans_files/figure-latex/unnamed-chunk-15-2.pdf}
\includegraphics{test_clust_kmeans_files/figure-latex/unnamed-chunk-15-3.pdf}

\hypertarget{test-14}{%
\subsubsection{Test 14}\label{test-14}}

Run the tests with:

\begin{itemize}
\tightlist
\item
  No PCA
\item
  min-aic k-means clustering. k is set to the number of pathways.
\item
  angles calculated before clustering.
\item
  number of pathways 4.
\item
  Centroids assigned using points
\end{itemize}

\begin{Shaded}
\begin{Highlighting}[]
\NormalTok{pc\_type }\OtherTok{\textless{}{-}} \StringTok{"No\_pca"}
\NormalTok{np }\OtherTok{\textless{}{-}} \DecValTok{3}
\NormalTok{d }\OtherTok{\textless{}{-}} \DecValTok{30}
\NormalTok{clust\_typ }\OtherTok{\textless{}{-}} \StringTok{"min"}
\NormalTok{space\_typ }\OtherTok{\textless{}{-}} \StringTok{"angle"}
\NormalTok{how\_cents }\OtherTok{\textless{}{-}} \StringTok{"points"}
\FunctionTok{test\_clust\_kmeans\_function}\NormalTok{(d,}
                           \AttributeTok{num\_path =}\NormalTok{ np,}
                           \AttributeTok{pc\_type =}\NormalTok{ pc\_type,}
                           \AttributeTok{clust\_typ =}\NormalTok{ clust\_typ,}
                           \AttributeTok{space\_typ =}\NormalTok{ space\_typ,}
                           \AttributeTok{how\_cents =}\NormalTok{ how\_cents)}
\end{Highlighting}
\end{Shaded}

\begin{verbatim}
## [1] "Test No_pca with 4 pathways"
##   bp_on clust_prob_on     clust_typ ndim_typ how_cents pc_type num_paths
## 1  TRUE          TRUE test_minangle test_all    points  No_pca         3
##           res_dir
## 1 PC_TestResults/
## [1] "Clusters converged 2"
## [1] "Clusters converged 7"
## [1] "Clusters converged 3"
## [1] "Clusters converged 3"
## [1] "Clusters converged 5"
\end{verbatim}

\includegraphics{test_clust_kmeans_files/figure-latex/unnamed-chunk-16-1.pdf}
\includegraphics{test_clust_kmeans_files/figure-latex/unnamed-chunk-16-2.pdf}
\includegraphics{test_clust_kmeans_files/figure-latex/unnamed-chunk-16-3.pdf}
\includegraphics{test_clust_kmeans_files/figure-latex/unnamed-chunk-16-4.pdf}

\hypertarget{test-15}{%
\subsubsection{Test 15}\label{test-15}}

Run the tests with:

\begin{itemize}
\tightlist
\item
  PCA with prcomp
\item
  min-aic k-means clustering. k is set to the number of pathways.
\item
  no angles calculated before clustering.
\item
  number of pathways 4.
\item
  Centroids assigned using points
\end{itemize}

\begin{Shaded}
\begin{Highlighting}[]
\NormalTok{pc\_type }\OtherTok{\textless{}{-}} \StringTok{"prcomp"}
\NormalTok{np }\OtherTok{\textless{}{-}} \DecValTok{3}
\NormalTok{d }\OtherTok{\textless{}{-}} \DecValTok{30}
\NormalTok{clust\_typ }\OtherTok{\textless{}{-}} \StringTok{"min"}
\NormalTok{space\_typ }\OtherTok{\textless{}{-}} \StringTok{"regular"}
\NormalTok{how\_cents }\OtherTok{\textless{}{-}} \StringTok{"points"}
\FunctionTok{test\_clust\_kmeans\_function}\NormalTok{(d,}
                           \AttributeTok{num\_path =}\NormalTok{ np,}
                           \AttributeTok{pc\_type =}\NormalTok{ pc\_type,}
                           \AttributeTok{clust\_typ =}\NormalTok{ clust\_typ,}
                           \AttributeTok{space\_typ =}\NormalTok{ space\_typ,}
                           \AttributeTok{how\_cents =}\NormalTok{ how\_cents)}
\end{Highlighting}
\end{Shaded}

\begin{verbatim}
## [1] "Test prcomp with 4 pathways"
##   bp_on clust_prob_on       clust_typ ndim_typ how_cents pc_type num_paths
## 1  TRUE          TRUE test_minregular test_all    points  prcomp         3
##           res_dir
## 1 PC_TestResults/
## [1] "Clusters converged 2"
## [1] "Clusters converged 6"
## [1] "Clusters converged 4"
## [1] "Clusters converged 5"
## [1] "Clusters converged 3"
\end{verbatim}

\includegraphics{test_clust_kmeans_files/figure-latex/unnamed-chunk-17-1.pdf}
\includegraphics{test_clust_kmeans_files/figure-latex/unnamed-chunk-17-2.pdf}
\includegraphics{test_clust_kmeans_files/figure-latex/unnamed-chunk-17-3.pdf}

\hypertarget{test-16}{%
\subsubsection{Test 16}\label{test-16}}

Run the tests with:

\begin{itemize}
\tightlist
\item
  PCA with prcomp
\item
  min-aic k-means clustering. k is set to the number of pathways.
\item
  angles calculated before PCA and clustering.
\item
  number of pathways 4.
\item
  Centroids assigned using points
\end{itemize}

\begin{Shaded}
\begin{Highlighting}[]
\NormalTok{pc\_type }\OtherTok{\textless{}{-}} \StringTok{"prcomp"}
\NormalTok{np }\OtherTok{\textless{}{-}} \DecValTok{3}
\NormalTok{d }\OtherTok{\textless{}{-}} \DecValTok{30}
\NormalTok{clust\_typ }\OtherTok{\textless{}{-}} \StringTok{"min"}
\NormalTok{space\_typ }\OtherTok{\textless{}{-}} \StringTok{"angle"}
\NormalTok{how\_cents }\OtherTok{\textless{}{-}} \StringTok{"points"}
\FunctionTok{test\_clust\_kmeans\_function}\NormalTok{(d,}
                           \AttributeTok{num\_path =}\NormalTok{ np,}
                           \AttributeTok{pc\_type =}\NormalTok{ pc\_type,}
                           \AttributeTok{clust\_typ =}\NormalTok{ clust\_typ,}
                           \AttributeTok{space\_typ =}\NormalTok{ space\_typ,}
                           \AttributeTok{how\_cents =}\NormalTok{ how\_cents)}
\end{Highlighting}
\end{Shaded}

\begin{verbatim}
## [1] "Test prcomp with 4 pathways"
##   bp_on clust_prob_on     clust_typ ndim_typ how_cents pc_type num_paths
## 1  TRUE          TRUE test_minangle test_all    points  prcomp         3
##           res_dir
## 1 PC_TestResults/
## [1] "Clusters converged 2"
## [1] "Clusters converged 6"
## [1] "Clusters converged 4"
## [1] "Clusters converged 5"
## [1] "Clusters converged 3"
\end{verbatim}

\includegraphics{test_clust_kmeans_files/figure-latex/unnamed-chunk-18-1.pdf}
\includegraphics{test_clust_kmeans_files/figure-latex/unnamed-chunk-18-2.pdf}
\includegraphics{test_clust_kmeans_files/figure-latex/unnamed-chunk-18-3.pdf}
\includegraphics{test_clust_kmeans_files/figure-latex/unnamed-chunk-18-4.pdf}

\end{document}
