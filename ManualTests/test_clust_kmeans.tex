% Options for packages loaded elsewhere
\PassOptionsToPackage{unicode}{hyperref}
\PassOptionsToPackage{hyphens}{url}
%
\documentclass[
]{article}
\usepackage{amsmath,amssymb}
\usepackage{iftex}
\ifPDFTeX
  \usepackage[T1]{fontenc}
  \usepackage[utf8]{inputenc}
  \usepackage{textcomp} % provide euro and other symbols
\else % if luatex or xetex
  \usepackage{unicode-math} % this also loads fontspec
  \defaultfontfeatures{Scale=MatchLowercase}
  \defaultfontfeatures[\rmfamily]{Ligatures=TeX,Scale=1}
\fi
\usepackage{lmodern}
\ifPDFTeX\else
  % xetex/luatex font selection
\fi
% Use upquote if available, for straight quotes in verbatim environments
\IfFileExists{upquote.sty}{\usepackage{upquote}}{}
\IfFileExists{microtype.sty}{% use microtype if available
  \usepackage[]{microtype}
  \UseMicrotypeSet[protrusion]{basicmath} % disable protrusion for tt fonts
}{}
\makeatletter
\@ifundefined{KOMAClassName}{% if non-KOMA class
  \IfFileExists{parskip.sty}{%
    \usepackage{parskip}
  }{% else
    \setlength{\parindent}{0pt}
    \setlength{\parskip}{6pt plus 2pt minus 1pt}}
}{% if KOMA class
  \KOMAoptions{parskip=half}}
\makeatother
\usepackage{xcolor}
\usepackage[margin=1in]{geometry}
\usepackage{longtable,booktabs,array}
\usepackage{calc} % for calculating minipage widths
% Correct order of tables after \paragraph or \subparagraph
\usepackage{etoolbox}
\makeatletter
\patchcmd\longtable{\par}{\if@noskipsec\mbox{}\fi\par}{}{}
\makeatother
% Allow footnotes in longtable head/foot
\IfFileExists{footnotehyper.sty}{\usepackage{footnotehyper}}{\usepackage{footnote}}
\makesavenoteenv{longtable}
\usepackage{graphicx}
\makeatletter
\def\maxwidth{\ifdim\Gin@nat@width>\linewidth\linewidth\else\Gin@nat@width\fi}
\def\maxheight{\ifdim\Gin@nat@height>\textheight\textheight\else\Gin@nat@height\fi}
\makeatother
% Scale images if necessary, so that they will not overflow the page
% margins by default, and it is still possible to overwrite the defaults
% using explicit options in \includegraphics[width, height, ...]{}
\setkeys{Gin}{width=\maxwidth,height=\maxheight,keepaspectratio}
% Set default figure placement to htbp
\makeatletter
\def\fps@figure{htbp}
\makeatother
\setlength{\emergencystretch}{3em} % prevent overfull lines
\providecommand{\tightlist}{%
  \setlength{\itemsep}{0pt}\setlength{\parskip}{0pt}}
\setcounter{secnumdepth}{-\maxdimen} % remove section numbering
\ifLuaTeX
  \usepackage{selnolig}  % disable illegal ligatures
\fi
\IfFileExists{bookmark.sty}{\usepackage{bookmark}}{\usepackage{hyperref}}
\IfFileExists{xurl.sty}{\usepackage{xurl}}{} % add URL line breaks if available
\urlstyle{same}
\hypersetup{
  pdftitle={Testing Clustering methods.},
  pdfauthor={Hayley Wragg},
  hidelinks,
  pdfcreator={LaTeX via pandoc}}

\title{Testing Clustering methods.}
\author{Hayley Wragg}
\date{21st Nov 2023}

\begin{document}
\maketitle

\hypertarget{description}{%
\section{Description}\label{description}}

The clustering program takes in GWAS data. This includes a matrix
\(\beta\) which contains the strength of association with a trait. The
trait forms the column and the snp is the row. There's also a matrix of
\(p\)-values these represented the probability that the observed
association is due to chance. There's also the \(SE\) matrix with the
standard error association with the association scores for each snp,
trait pair.

The snp's are chosen based on those with the strongest association to
the exposure.

The algorithm attempts to use the association with different traits to
identify separate pathways within the data. We expect the pathways to
appear as lines in the original data-space, in order to identify
clusters we use two methods to reformat the data. The first is
transforming into the space given by the angles to the axes, this should
take lines through the origin with different gradients to balls of data.
The second is principal component analysis, which identifies to the
strongest trends within the data. In this testing process we will look
at different combinations of these.

For the test cases I have created synthetics which intentionally has
pathways in.

\hypertarget{test-case-table}{%
\subsubsection{Test Case Table}\label{test-case-table}}

\begin{longtable}[]{@{}
  >{\raggedright\arraybackslash}p{(\columnwidth - 12\tabcolsep) * \real{0.1429}}
  >{\raggedright\arraybackslash}p{(\columnwidth - 12\tabcolsep) * \real{0.1429}}
  >{\raggedright\arraybackslash}p{(\columnwidth - 12\tabcolsep) * \real{0.1429}}
  >{\raggedright\arraybackslash}p{(\columnwidth - 12\tabcolsep) * \real{0.1429}}
  >{\raggedright\arraybackslash}p{(\columnwidth - 12\tabcolsep) * \real{0.1429}}
  >{\raggedright\arraybackslash}p{(\columnwidth - 12\tabcolsep) * \real{0.1429}}
  >{\raggedright\arraybackslash}p{(\columnwidth - 12\tabcolsep) * \real{0.1429}}@{}}
\toprule\noalign{}
\begin{minipage}[b]{\linewidth}\raggedright
Test No.
\end{minipage} & \begin{minipage}[b]{\linewidth}\raggedright
No.~of pathways
\end{minipage} & \begin{minipage}[b]{\linewidth}\raggedright
\protect\hyperlink{pca}{PCA method}
\end{minipage} & \begin{minipage}[b]{\linewidth}\raggedright
\protect\hyperlink{angles}{Angles used (Y or N)}
\end{minipage} & \begin{minipage}[b]{\linewidth}\raggedright
\protect\hyperlink{cluster_p_weight}{Weight cluster contribution with
p-values.}
\end{minipage} & \begin{minipage}[b]{\linewidth}\raggedright
Clustering method
\end{minipage} & \begin{minipage}[b]{\linewidth}\raggedright
\protect\hyperlink{how_cents}{Centroid assignment}
\end{minipage} \\
\midrule\noalign{}
\endhead
\bottomrule\noalign{}
\endlastfoot
\protect\hyperlink{test1}{1} & 2 & none & N & N & standard k-means &
random \\
\protect\hyperlink{test2}{2} & \textbf{4} & none & N & N & standard
k-means & random \\
\protect\hyperlink{test3}{3} & 2 & none & \textbf{Y} & N & standard
k-means & random \\
\protect\hyperlink{test4}{4} & \textbf{4} & none & Y & N & standard
k-means & random \\
\protect\hyperlink{test6}{6} & 4 & none & Y & N & standard k-means &
\textbf{random from points} \\
\protect\hyperlink{test7}{7} & 4 & \textbf{prcomp} & N & N & standard
k-means & random from points \\
\protect\hyperlink{test8}{8} & 4 & prcomp & \textbf{Y} & N & standard
k-means & random from points \\
\protect\hyperlink{test9}{9} & 4 & prcomp & Y & \textbf{Y} & standard
k-means & random from points \\
\protect\hyperlink{test10}{10} & 4 & \textbf{none} & \textbf{N} &
\textbf{N} & \protect\hyperlink{min_aic}{\textbf{min-aic k-means}} &
random from points \\
\protect\hyperlink{test11}{11} & 4 & none & Y & N & \textbf{min-aic
k-means} & random from points \\
\protect\hyperlink{test12}{12} & 4 & none & Y & \textbf{Y} & min-aic
k-means & random from points \\
\protect\hyperlink{test13}{13} & 4 & \textbf{prcomp} & \textbf{N} &
\textbf{N} & min-aic k-means & random from points \\
\protect\hyperlink{test14}{14} & 4 & prcomp & \textbf{Y} & N & min-aic
k-means & random from points \\
\protect\hyperlink{test15}{15} & 4 & prcomp & Y & \textbf{Y} & min-aic
k-means & random from points \\
\protect\hyperlink{test16}{10} & 4 & \textbf{none} & \textbf{N} &
\textbf{N} & \protect\hyperlink{dbscan}{\textbf{dbscan}} & centroids not
assigned in density based methods \\
\protect\hyperlink{test17}{11} & 4 & none & Y & N & \textbf{dbscan} &
centroids not assigned in density based methods \\
\protect\hyperlink{test18}{13} & 4 & \textbf{prcomp} & \textbf{N} &
\textbf{N} & dbscan & centroids not assigned in density based methods \\
\protect\hyperlink{test19}{14} & 4 & prcomp & \textbf{Y} & N & dbscan &
centroids not assigned in density based methods \\
\protect\hyperlink{test20}{15} & 4 & prcomp & Y & N & dbscan & centroids
not assigned in density based methods \\
\protect\hyperlink{test20}{15} & 4 & prcomp & Y & N & dbscan & centroids
not assigned in density based methods \\
\end{longtable}

\hypertarget{instructions-for-using-the-package.}{%
\section{Instructions for using the
package.}\label{instructions-for-using-the-package.}}

\hypertarget{install-package}{%
\subsection{Install package}\label{install-package}}

Load the ``ndimclusteringR'' package

\hypertarget{iteration-parameters}{%
\subsection{Iteration parameters}\label{iteration-parameters}}

The parameters for the program are:

\begin{itemize}
\item
  \texttt{dname} The directory where the data is saved. This should
  include the following files:

  \begin{itemize}
  \item
    \texttt{unstdBeta\_df.csv} A \texttt{.csv} file with columns
    seperated by \texttt{,} and rows by \texttt{;} , each row
    corresponds to a snp and each column is a trait. The values
    correspond to the intensity of association determined from a GWAS.
    The data is sourced from openGWAS.
  \item
    \texttt{unstdSE\_df.csv} A \texttt{.csv} file with columns seperated
    by \texttt{,} and rows by \texttt{;} , each row corresponds to a snp
    and each column is a trait. The values correspond to the standard
    error of the association determined from a GWAS. The data is sourced
    from openGWAS.
  \item
    \texttt{pval\_df.csv} A \texttt{.csv} file with columns seperated by
    \texttt{,} and rows by \texttt{;} , each row corresponds to a snp
    and each column is a trait. The values correspond to the
    \(p\)-values of the association determined from a GWAS. The
    \(p\)-values are the probability that the intensity values are
    observed due to chance. The data is sourced from openGWAS.
  \item
    \texttt{trait\_info\_nfil.csv} A \texttt{.csv} file with columns
    seperated by \texttt{,} and rows by \texttt{;} , each row
    corresponds to a trait. The label for the trait is
    \texttt{phenotype}, to get the full definition for the trait as used
    in the data collection this is found in the column
    \texttt{description}.
  \end{itemize}
\item
  \texttt{clust\_type} method of clustering to be used.
\item
  \texttt{basic} is a standard \(k\)-means clustering method.
\item
  \texttt{min} is an iteration of \(k\)-means, which minimises the
  \(aic\) from cluster sets run from \(1\) cluster to \(k\).
\item
  \texttt{dbscan} density based clustering method.
\item
  \texttt{pca\_type} method for determining the principle components.
  See \protect\hyperlink{pca}{PCA section}

  \begin{itemize}
  \tightlist
  \item
    \texttt{prcomp} PCA is performed using the \texttt{prcomp} package.
  \item
    \texttt{none} no PCA is performed and the data is clustered without
    PCA.
  \end{itemize}
\item
  \texttt{nclust} The number of clusters to search for when using
  \texttt{basic} or \texttt{min} clustering.
\item
  \texttt{n\_pcs} The number of principal components to reduce to. This
  term is only used if \texttt{pca\_type} is NOT \texttt{none}.
\item
  \texttt{bin\_angles} Binary switch determining whether the data should
  be transformed to the angle space or not. If \(0\) then nothing is
  done, if \(1\) then the data is transformed from the intensity scores
  to the angle of those scores to the axis in that data-space. Since the
  angle describe a position betwen two axes by using the angles we can
  reduce the dimension of the data-space by \(1\), if we wanted to
  retained all information we would also store the distance from the
  origin but we are searching for elements on the same line, it does not
  matter to our clusters where on the line they are, therefore we do not
  store this distance.
\item
  \texttt{bin\_p\_clust} Binary switch. If \(0\) then \(k\)-means
  clustering calculated the cluster centroids based on the average of
  the terms in that cluster. If \(1\) then the cluster centroids are
  recalculated based on a weighted average using the intensity and the
  \(p\)-values. See \protect\hyperlink{cluster_p_weight}{\(p\)-value
  weighting section}.
\item
  \texttt{bin\_p\_score} Binary switch indicating the weighting when
  scoring clusters on trait axis. If \(0\) then the clusters are scored
  based on their elements average intensity scores on each axis. If
  \(1\) then the clusters are scored based on a weight average of their
  elements scores on each trait, weighted by the \(p\)-value that the
  intensity score is a true representation.
\item
  \texttt{how\_cents} The method to be used for initialising the centres
  of the clusters.

  \begin{itemize}
  \tightlist
  \item
    \texttt{rand} The value on each axis is randomly assigned using a
    uniform distribution between the max and min on each axis.
  \item
    \texttt{points} The centre for each cluster is assigned to points
    from the dataset. The points are randomly selected with a uniform
    distribution.
  \end{itemize}
\end{itemize}

\hypertarget{alg_steps}{%
\subsubsection{What happens within the program? - Algorithm: Step by
step}\label{alg_steps}}

If all the steps are included then they are performed in the following
order.

\begin{enumerate}
\def\labelenumi{\arabic{enumi}.}
\tightlist
\item
  Crop data to complete cases.
\item
  Convert data to angles.
\item
  Perform PCA on angle data.
\item
  Cluster on the PCA
\end{enumerate}

\begin{itemize}
\tightlist
\item
  Set cluster centres
\item
  Converge cluster centres
\item
  if cluster type ``min'' then rerun for different number of clusters
  and choose set with min-aic.
\end{itemize}

\hypertarget{test1}{%
\subsubsection{Test 1}\label{test1}}

Run the tests with:

\begin{itemize}
\tightlist
\item
  No PCA
\item
  basic k-means clustering.

  \begin{itemize}
  \tightlist
  \item
    k is set to the number of pathways.
  \end{itemize}
\item
  no angles calculated before clustering.
\item
  number of pathways 2.
\end{itemize}

\begin{verbatim}
## [1] "Clusters converged 3"
\end{verbatim}

\includegraphics{test_clust_kmeans_files/figure-latex/unnamed-chunk-3-1.pdf}
\includegraphics{test_clust_kmeans_files/figure-latex/unnamed-chunk-3-2.pdf}

\hypertarget{test2}{%
\subsubsection{Test 2}\label{test2}}

For test2 we will keep the parameters the same as
\protect\hyperlink{test1}{test1} but increase the number of pathways.

Run the tests with:

\begin{itemize}
\tightlist
\item
  No PCA
\item
  basic k-means clustering.

  \begin{itemize}
  \tightlist
  \item
    k is set to the number of pathways.
  \end{itemize}
\item
  no angles calculated before clustering.
\item
  \textbf{number of pathways 4.}
\item
  Clusters centres are not weighted by the p-values.
\end{itemize}

\begin{verbatim}
## [1] "Clusters converged 5"
\end{verbatim}

\includegraphics{test_clust_kmeans_files/figure-latex/unnamed-chunk-5-1.pdf}
\includegraphics{test_clust_kmeans_files/figure-latex/unnamed-chunk-5-2.pdf}

\hypertarget{angles}{%
\subsection{Calculating Angles}\label{angles}}

The data-points for each snp have an intensity score associated with
each trait. We want to identify lines within our data but we can see
from the first two tests that standard clustering does not detect these
lines. To try and convert data that is visualised as lines to something
that looks like balls of data we convert the data to the angle to each
axis.

\begin{enumerate}
\def\labelenumi{\arabic{enumi}.}
\tightlist
\item
  Create a unit-vector for each axis (you can remove one axis since the
  angles reduce the dimension), the algorithm removes the last axis.
\item
  For each unit vector found and each data-point \(x\):

  \begin{itemize}
  \tightlist
  \item
    \[\theta = \frac{x \cdot u}{||u|| ||x||}\]
  \item
    If \(\theta > \pi\) reassign the angle \(\theta= \theta-\pi\). These
    are angles on opposite sides of the circle but they lie on the same
    line so we want the algorithm to group them together.
  \end{itemize}
\end{enumerate}

\hypertarget{test3}{%
\subsubsection{Test 3}\label{test3}}

For test3 we will return to the case of two pathways but this time we
will reformat the data before clustering. We will transform the data
from the raw \(\beta\) intensity scores to the angles of the scores when
transformed to a polar-coordinate system.

Run the tests with:

\begin{itemize}
\tightlist
\item
  No PCA
\item
  basic k-means clustering.

  \begin{itemize}
  \tightlist
  \item
    k is set to the number of pathways.
  \end{itemize}
\item
  \textbf{angles calculated.}
\item
  number of pathways 2.
\end{itemize}

\begin{verbatim}
## [1] "Clusters converged 4"
\end{verbatim}

\includegraphics{test_clust_kmeans_files/figure-latex/unnamed-chunk-7-1.pdf}
\includegraphics{test_clust_kmeans_files/figure-latex/unnamed-chunk-7-2.pdf}
\includegraphics{test_clust_kmeans_files/figure-latex/unnamed-chunk-7-3.pdf}
\includegraphics{test_clust_kmeans_files/figure-latex/unnamed-chunk-7-4.pdf}

\hypertarget{test4}{%
\subsubsection{Test 4}\label{test4}}

For test4 we use the angles again but we investigate the case with 4
pathways.

Run the tests with:

\begin{itemize}
\tightlist
\item
  No PCA
\item
  basic k-means clustering.

  \begin{itemize}
  \tightlist
  \item
    k is set to the number of pathways.
  \end{itemize}
\item
  angles calculated before clustering.
\item
  \textbf{number of pathways 4.}
\end{itemize}

\begin{verbatim}
## [1] "Clusters converged 3"
\end{verbatim}

\includegraphics{test_clust_kmeans_files/figure-latex/unnamed-chunk-9-1.pdf}
\includegraphics{test_clust_kmeans_files/figure-latex/unnamed-chunk-9-2.pdf}
\includegraphics{test_clust_kmeans_files/figure-latex/unnamed-chunk-9-3.pdf}
\includegraphics{test_clust_kmeans_files/figure-latex/unnamed-chunk-9-4.pdf}

\hypertarget{how_cents}{%
\subsection{Consider the location of the centroids.}\label{how_cents}}

The previous tests used centroids assigned randomly within the ranges on
each axis. The results show that although we ask for \(k\) clusters we
often get some of the clusters returning as empty. This is likely to be
caused by the initial centroids being outside the data-space.

Instead we will initialise the centroids using randomly selected points
from out dataset.

\hypertarget{test5}{%
\subsubsection{Test 5}\label{test5}}

For test5 we keep the parameters the same as
\protect\hyperlink{test2}{test2} but we change the initialisation of the
cluster centroids.

Run the tests with:

\begin{itemize}
\tightlist
\item
  No PCA
\item
  basic k-means clustering.

  \begin{itemize}
  \tightlist
  \item
    k is set to the number of pathways.
  \end{itemize}
\item
  no angles calculated before clustering.
\item
  number of pathways 4.
\item
  \textbf{Centroids assigned using points.}
\end{itemize}

\begin{verbatim}
## [1] "Clusters converged 6"
\end{verbatim}

\includegraphics{test_clust_kmeans_files/figure-latex/unnamed-chunk-11-1.pdf}
\includegraphics{test_clust_kmeans_files/figure-latex/unnamed-chunk-11-2.pdf}

\hypertarget{test6}{%
\subsubsection{Test 6}\label{test6}}

For test6 we keep the parameters the same as
\protect\hyperlink{test5}{test5} but we include the conversion to
angles.

Run the tests with:

\begin{itemize}
\tightlist
\item
  No PCA
\item
  basic k-means clustering.

  \begin{itemize}
  \tightlist
  \item
    k is set to the number of pathways.
  \end{itemize}
\item
  \textbf{angles calculated before clustering.}
\item
  number of pathways 4.
\item
  Centroids assigned using points
\end{itemize}

\begin{verbatim}
## [1] "Clusters converged 2"
\end{verbatim}

\includegraphics{test_clust_kmeans_files/figure-latex/unnamed-chunk-13-1.pdf}
\includegraphics{test_clust_kmeans_files/figure-latex/unnamed-chunk-13-2.pdf}
\includegraphics{test_clust_kmeans_files/figure-latex/unnamed-chunk-13-3.pdf}
\includegraphics{test_clust_kmeans_files/figure-latex/unnamed-chunk-13-4.pdf}

\hypertarget{pca}{%
\subsection{Principal Components}\label{pca}}

So far we have ignored the principal components option. This is another
method for reducing the dimension. It identifies the eigen-vectors and
corresponding eigen-values of the data space. Then outputs the set of
eigen-vectors with the largest eigen-values as a matrix. This matrix is
then used to transform the data onto a new data-space with reduced
dimension. The principal components characterise the variation in the
data. By defining the axes as the eigen-vectors we remove the highest
levels of correlation within the data. In doing this we are trying to
find the level at which the data separates into distinct categories.

PCA steps:

\begin{enumerate}
\def\labelenumi{\arabic{enumi}.}
\tightlist
\item
  Ensure each data axis is normalised.
\item
  Identify the eigen-vectors and eigen-values of the data-space.
\item
  Use the eigen-vectors of the \texttt{n\_pc} largest eigen-values to
  form the columns of a matrix.
\item
  Take each data-point as a row-vector and multiply by the matrix to
  transform the data onto a new data-space.
\item
  Transform the p-value matrix and standard error matrix onto the same
  dataspace.
\item
  Store the transform matrix. This represents the proportion of each
  trait which makes up each pc vector.
\end{enumerate}

\href{https://www.rdocumentation.org/packages/stats/versions/3.6.2/topics/prcomp}{\texttt{prcomp}}
is a pca package in R.

\hypertarget{test7}{%
\subsubsection{Test 7}\label{test7}}

For the first pca test we will turn off the calculation of angles and
use a standard k-means clustering to try and visualise what the pca is
doing on it's own. This is \protect\hyperlink{test5}{test5} with
\texttt{how\_cents} changed.

Run the tests with:

\begin{itemize}
\tightlist
\item
  \textbf{PCA method prcomp}
\item
  basic k-means clustering.

  \begin{itemize}
  \tightlist
  \item
    k is set to the number of pathways.
  \end{itemize}
\item
  no angles calculated before clustering.
\item
  number of pathways 4.
\item
  Centroids assigned using points
\end{itemize}

\begin{verbatim}
## [1] "Clusters converged 4"
\end{verbatim}

\includegraphics{test_clust_kmeans_files/figure-latex/unnamed-chunk-15-1.pdf}
\includegraphics{test_clust_kmeans_files/figure-latex/unnamed-chunk-15-2.pdf}

\hypertarget{test8}{%
\subsubsection{Test 8}\label{test8}}

For the second pca test we will include the calculation of angles and
use a standard k-means clustering to try and visualise what how the pca
works with the angle data. This is \protect\hyperlink{test6}{test6} with
\texttt{how\_cents} changed.

Run the tests with:

\begin{itemize}
\tightlist
\item
  PCA method prcomp
\item
  basic k-means clustering.

  \begin{itemize}
  \tightlist
  \item
    k is set to the number of pathways.
  \end{itemize}
\item
  \textbf{angles calculated before clustering.}
\item
  number of pathways 4.
\item
  Centroids assigned using points
\end{itemize}

\begin{verbatim}
## [1] "Clusters converged 7"
\end{verbatim}

\includegraphics{test_clust_kmeans_files/figure-latex/unnamed-chunk-17-1.pdf}
\includegraphics{test_clust_kmeans_files/figure-latex/unnamed-chunk-17-2.pdf}
\includegraphics{test_clust_kmeans_files/figure-latex/unnamed-chunk-17-3.pdf}
\includegraphics{test_clust_kmeans_files/figure-latex/unnamed-chunk-17-4.pdf}

\hypertarget{cluster_p_weight}{%
\subsection{Weight cluster centroid by p-value}\label{cluster_p_weight}}

Each data-point has an associated p-value. This corresponds to the
chance the score observed is due to chance. We do not want to skew our
conclusions by unreliable snps. Therefore when the cluster centroids are
found we will use a weighted average based on the p-value. For each
co-ordinate in the cluster centroid the new centroid after cluster
assignment is given by:

\[
\pmb{c}_i = \frac{\sum_{\text{points in cluster}}\beta_i(1-p_i)}{\sum_{\text{points in cluster}}(1-p_i)}
\]

\hypertarget{test9}{%
\subsubsection{Test 9}\label{test9}}

Run the tests with:

\begin{itemize}
\tightlist
\item
  PCA method prcomp
\item
  basic k-means clustering.

  \begin{itemize}
  \tightlist
  \item
    k is set to the number of pathways.
  \end{itemize}
\item
  angles calculated before clustering.
\item
  number of pathways 4.
\item
  Centroids assigned using points
\item
  \textbf{Centroids reassigned during clustering using a p-value
  weighting.}
\end{itemize}

\begin{verbatim}
## [1] "Clusters converged 1"
\end{verbatim}

\includegraphics{test_clust_kmeans_files/figure-latex/unnamed-chunk-19-1.pdf}
\includegraphics{test_clust_kmeans_files/figure-latex/unnamed-chunk-19-2.pdf}
\includegraphics{test_clust_kmeans_files/figure-latex/unnamed-chunk-19-3.pdf}
\includegraphics{test_clust_kmeans_files/figure-latex/unnamed-chunk-19-4.pdf}

\hypertarget{min-aic}{%
\subsection{Min-AIC}\label{min-aic}}

So far we have clustered using a fixed number of clusters. Since we are
using synthetic test data we can set the value for \(k\) to the number
of clusters we expect. However when data-mining we don't know how many
clusters we are looking for. We consider an iterative clustering
approach, running a standard k-means clustering on \(1\) to \(k\)
clusters. For each set of clusters calculate the Akaike Information
Criterion (AIC), then proceed with the cluster set that minimises this.

The AIC is given by: \[ 
l = \frac{\sum_{\text{data_points}}\text{dist}\left(\beta_i, c_{\text{centre for cluster assigned to } \beta_i}\right)^2}{\text{var}(\text{dist}\left(\pmb{\beta}\right))}, \\
d = \text{dimension of each point},\\
k = \text{number of points}, \\
AIC = d + 2  k  l
\]

For \(nc\) in \([1,k]\)

\begin{enumerate}
\def\labelenumi{\arabic{enumi}.}
\tightlist
\item
  Set the number of clusters to \(nc\).

  \begin{itemize}
  \tightlist
  \item
    If \(nc>k\) go to step 6.
  \end{itemize}
\item
  Run k-means clustering for \(k=nc\).
\item
  Calculate the \(aic\) for the clusters.
\item
  Increase \(nc\) by \(1\).
\item
  Go to step 1.
\item
  Find the number of clusters which gives the minimum aic.
\item
  Return the cluster set for this number of clusters.
\end{enumerate}

We will perform the following tests on the min-aic clustering method:

\begin{longtable}[]{@{}
  >{\raggedright\arraybackslash}p{(\columnwidth - 12\tabcolsep) * \real{0.1370}}
  >{\raggedright\arraybackslash}p{(\columnwidth - 12\tabcolsep) * \real{0.1370}}
  >{\raggedright\arraybackslash}p{(\columnwidth - 12\tabcolsep) * \real{0.1370}}
  >{\raggedright\arraybackslash}p{(\columnwidth - 12\tabcolsep) * \real{0.1370}}
  >{\raggedright\arraybackslash}p{(\columnwidth - 12\tabcolsep) * \real{0.1781}}
  >{\raggedright\arraybackslash}p{(\columnwidth - 12\tabcolsep) * \real{0.1370}}
  >{\raggedright\arraybackslash}p{(\columnwidth - 12\tabcolsep) * \real{0.1370}}@{}}
\toprule\noalign{}
\begin{minipage}[b]{\linewidth}\raggedright
Test No.
\end{minipage} & \begin{minipage}[b]{\linewidth}\raggedright
No.~of pathways
\end{minipage} & \begin{minipage}[b]{\linewidth}\raggedright
\protect\hyperlink{pca}{PCA method}
\end{minipage} & \begin{minipage}[b]{\linewidth}\raggedright
\protect\hyperlink{angles}{Angles used (Y or N)}
\end{minipage} & \begin{minipage}[b]{\linewidth}\raggedright
\protect\hyperlink{cluster_p_weight}{Weight cluster contribution with
p-values.}
\end{minipage} & \begin{minipage}[b]{\linewidth}\raggedright
Clustering method
\end{minipage} & \begin{minipage}[b]{\linewidth}\raggedright
\protect\hyperlink{how_cents}{Centroid assignment}
\end{minipage} \\
\midrule\noalign{}
\endhead
\bottomrule\noalign{}
\endlastfoot
\protect\hyperlink{test10}{10} & 4 & \textbf{none} & \textbf{N} &
\textbf{N} & \protect\hyperlink{min_aic}{\textbf{min-aic k-means}} &
random from points \\
\protect\hyperlink{test11}{11} & 4 & none & Y & N & \textbf{min-aic
k-means} & random from points \\
\protect\hyperlink{test12}{12} & 4 & none & Y & \textbf{Y} & min-aic
k-means & random from points \\
\protect\hyperlink{test13}{13} & 4 & \textbf{prcomp} & \textbf{N} &
\textbf{N} & min-aic k-means & random from points \\
\protect\hyperlink{test14}{14} & 4 & prcomp & \textbf{Y} & N & min-aic
k-means & random from points \\
\protect\hyperlink{test15}{15} & 4 & prcomp & Y & \textbf{Y} & min-aic
k-means & random from points \\
\end{longtable}

\hypertarget{test10}{%
\subsubsection{Test 10}\label{test10}}

Run the tests with:

\begin{itemize}
\tightlist
\item
  No PCA
\item
  \textbf{min-aic k-means clustering.}

  \begin{itemize}
  \tightlist
  \item
    Max \(k\) is set to the number of pathways.
  \end{itemize}
\item
  no angles calculated before clustering.
\item
  number of pathways 4.
\item
  Centroids assigned using points
\item
  Centroids are not reassigned during clustering using a p-value
  weighting.
\end{itemize}

\begin{verbatim}
## [1] "Clusters converged 2"
## [1] "Clusters converged 2"
## [1] "Clusters converged 5"
## [1] "Clusters converged 3"
\end{verbatim}

\includegraphics{test_clust_kmeans_files/figure-latex/unnamed-chunk-21-1.pdf}
\includegraphics{test_clust_kmeans_files/figure-latex/unnamed-chunk-21-2.pdf}

\hypertarget{test11}{%
\subsubsection{Test 11}\label{test11}}

For test 11 we keep everything the same as
\protect\hyperlink{test10}{test10} with the addition of the conversion
to angles before clustering.

Run the tests with:

\begin{itemize}
\tightlist
\item
  No PCA
\item
  min-aic k-means clustering.

  \begin{itemize}
  \tightlist
  \item
    Max \(k\) is set to the number of pathways.
  \end{itemize}
\item
  \textbf{angles calculated before clustering.}
\item
  number of pathways 4.
\item
  Centroids assigned using points
\item
  Centroids are not reassigned during clustering using a p-value
  weighting.
\end{itemize}

\begin{verbatim}
## [1] "Clusters converged 2"
## [1] "Clusters converged 2"
## [1] "Clusters converged 2"
## [1] "Clusters converged 5"
\end{verbatim}

\includegraphics{test_clust_kmeans_files/figure-latex/unnamed-chunk-23-1.pdf}
\includegraphics{test_clust_kmeans_files/figure-latex/unnamed-chunk-23-2.pdf}
\includegraphics{test_clust_kmeans_files/figure-latex/unnamed-chunk-23-3.pdf}
\includegraphics{test_clust_kmeans_files/figure-latex/unnamed-chunk-23-4.pdf}

\hypertarget{test12}{%
\subsubsection{Test 12}\label{test12}}

For test 12 we keep everything the same as
\protect\hyperlink{test11}{test11} with the addition of weighting the
cluster centre's by the p-values.

Run the tests with:

\begin{itemize}
\tightlist
\item
  No PCA
\item
  min-aic k-means clustering.

  \begin{itemize}
  \tightlist
  \item
    Max \(k\) is set to the number of pathways.
  \end{itemize}
\item
  angles calculated before clustering.
\item
  number of pathways 4.
\item
  Centroids assigned using points
\item
  \textbf{Centroids are reassigned during clustering using a p-value
  weighting.}
\end{itemize}

\begin{verbatim}
## [1] "Clusters converged 1"
## [1] "Clusters converged 1"
## [1] "Clusters converged 1"
## [1] "Clusters converged 1"
\end{verbatim}

\includegraphics{test_clust_kmeans_files/figure-latex/unnamed-chunk-25-1.pdf}
\includegraphics{test_clust_kmeans_files/figure-latex/unnamed-chunk-25-2.pdf}
\includegraphics{test_clust_kmeans_files/figure-latex/unnamed-chunk-25-3.pdf}
\includegraphics{test_clust_kmeans_files/figure-latex/unnamed-chunk-25-4.pdf}

\hypertarget{test13}{%
\subsubsection{Test 13}\label{test13}}

For test 13 we return to the set up of the first min-aic test in
\protect\hyperlink{test10}{test10} with the addition of principal
component analysis.

Run the tests with:

\begin{itemize}
\tightlist
\item
  \textbf{PCA using prcomp}
\item
  min-aic k-means clustering.

  \begin{itemize}
  \tightlist
  \item
    Max \(k\) is set to the number of pathways.
  \end{itemize}
\item
  no angles calculated before clustering.
\item
  number of pathways 4.
\item
  Centroids assigned using points
\item
  Centroids are not reassigned during clustering using a p-value
  weighting.
\end{itemize}

\begin{verbatim}
## [1] "Clusters converged 2"
## [1] "Clusters converged 3"
## [1] "Clusters converged 2"
## [1] "Clusters converged 3"
\end{verbatim}

\includegraphics{test_clust_kmeans_files/figure-latex/unnamed-chunk-27-1.pdf}
\includegraphics{test_clust_kmeans_files/figure-latex/unnamed-chunk-27-2.pdf}

\hypertarget{test14}{%
\subsubsection{Test 14}\label{test14}}

For test 14 we add the angle conversion to
\protect\hyperlink{test13}{test13}.

Run the tests with:

\begin{itemize}
\tightlist
\item
  PCA using prcomp
\item
  min-aic k-means clustering.

  \begin{itemize}
  \tightlist
  \item
    Max \(k\) is set to the number of pathways.
  \end{itemize}
\item
  \textbf{angles calculated before clustering.}
\item
  number of pathways 4.
\item
  Centroids assigned using points
\item
  Centroids are not reassigned during clustering using a p-value
  weighting.
\end{itemize}

\begin{verbatim}
## [1] "Clusters converged 2"
## [1] "Clusters converged 4"
## [1] "Clusters converged 5"
## [1] "Clusters converged 5"
\end{verbatim}

\includegraphics{test_clust_kmeans_files/figure-latex/unnamed-chunk-29-1.pdf}
\includegraphics{test_clust_kmeans_files/figure-latex/unnamed-chunk-29-2.pdf}
\includegraphics{test_clust_kmeans_files/figure-latex/unnamed-chunk-29-3.pdf}
\includegraphics{test_clust_kmeans_files/figure-latex/unnamed-chunk-29-4.pdf}

\hypertarget{test15}{%
\subsubsection{Test 15}\label{test15}}

For test 15 we take the set up in \protect\hyperlink{test14}{test14} and
we \protect\hyperlink{cluster_p_weight}{weight the cluster centre's
using the p-value weighting}.

Run the tests with:

\begin{itemize}
\tightlist
\item
  PCA using prcomp
\item
  min-aic k-means clustering.

  \begin{itemize}
  \tightlist
  \item
    Max \(k\) is set to the number of pathways.
  \end{itemize}
\item
  angles calculated before clustering.
\item
  number of pathways 4.
\item
  Centroids assigned using points
\item
  \protect\hyperlink{cluster_p_weight}{\textbf{Centroids are reassigned
  during clustering using a p-value weighting.}}
\end{itemize}

\begin{verbatim}
## [1] "Clusters converged 1"
## [1] "Clusters converged 1"
## [1] "Clusters converged 1"
## [1] "Clusters converged 1"
\end{verbatim}

\includegraphics{test_clust_kmeans_files/figure-latex/unnamed-chunk-31-1.pdf}
\includegraphics{test_clust_kmeans_files/figure-latex/unnamed-chunk-31-2.pdf}
\includegraphics{test_clust_kmeans_files/figure-latex/unnamed-chunk-31-3.pdf}
\includegraphics{test_clust_kmeans_files/figure-latex/unnamed-chunk-31-4.pdf}

\hypertarget{dbscan}{%
\subsection{DBSCAN}\label{dbscan}}

\textbf{dbscan} is a density based clustering method. Points are
assigned to the cluster if there are within range of another point in
the cluster. As a result the clusters can be different shapes and not
necessarily balls.

Dbscan does not require a predefined number of clusters, it instead
takes in a set density for the points \texttt{point\_eps}.

We will repeat the tests we used for \protect\hyperlink{minaic}{min-aic}
changing the clustering method to dbscan, however we won't vary the
centroid calculations as centroids aren't used in density clustering.

\hypertarget{test16}{%
\subsubsection{Test 16}\label{test16}}

Run the tests with:

\begin{itemize}
\tightlist
\item
  No PCA
\item
  \textbf{dbscan clustering.}
\item
  no angles calculated before clustering.
\item
  number of pathways 4.
\item
  Centroids not assigned in density based methods.
\end{itemize}

\includegraphics{test_clust_kmeans_files/figure-latex/unnamed-chunk-33-1.pdf}
\includegraphics{test_clust_kmeans_files/figure-latex/unnamed-chunk-33-2.pdf}

\hypertarget{test17}{%
\subsubsection{Test 17}\label{test17}}

Take the case from \protect\hyperlink{test16}{test 16} and introduce the
conversion to angles before clustering.

Run the tests with:

\begin{itemize}
\tightlist
\item
  No PCA
\item
  dbscan clustering.
\item
  \textbf{Angles calculated before clustering.}
\item
  number of pathways 4.
\item
  Centroids not assigned in density based methods.
\end{itemize}

\includegraphics{test_clust_kmeans_files/figure-latex/unnamed-chunk-35-1.pdf}
\includegraphics{test_clust_kmeans_files/figure-latex/unnamed-chunk-35-2.pdf}
\includegraphics{test_clust_kmeans_files/figure-latex/unnamed-chunk-35-3.pdf}
\includegraphics{test_clust_kmeans_files/figure-latex/unnamed-chunk-35-4.pdf}

\hypertarget{test18}{%
\subsubsection{Test 18}\label{test18}}

Take the case from \protect\hyperlink{test16}{test 16} and introduce the
conversion to principal components before clustering. First we will do
this without calculating the angles.

Run the tests with:

\begin{itemize}
\tightlist
\item
  No PCA
\item
  dbscan clustering.
\item
  \textbf{Angles calculated before clustering.}
\item
  number of pathways 4.
\item
  Centroids not assigned in density based methods.
\end{itemize}

\includegraphics{test_clust_kmeans_files/figure-latex/unnamed-chunk-37-1.pdf}
\includegraphics{test_clust_kmeans_files/figure-latex/unnamed-chunk-37-2.pdf}

\hypertarget{test19}{%
\subsubsection{Test 19}\label{test19}}

Now let's combine \protect\hyperlink{test17}{test 17} and
\protect\hyperlink{test18}{test 18} and calculate the principal
components and the angles before clustering.

Run the tests with:

\begin{itemize}
\tightlist
\item
  \textbf{PCA calculated using prcomp}
\item
  dbscan clustering.
\item
  \textbf{Angles calculated before clustering.}
\item
  number of pathways 4.
\item
  Centroids not assigned in density based methods.
\end{itemize}

\includegraphics{test_clust_kmeans_files/figure-latex/unnamed-chunk-39-1.pdf}
\includegraphics{test_clust_kmeans_files/figure-latex/unnamed-chunk-39-2.pdf}
\includegraphics{test_clust_kmeans_files/figure-latex/unnamed-chunk-39-3.pdf}
\includegraphics{test_clust_kmeans_files/figure-latex/unnamed-chunk-39-4.pdf}

\hypertarget{vary-the-density}{%
\subsection{Vary the density}\label{vary-the-density}}

\hypertarget{test20}{%
\subsubsection{Test 20}\label{test20}}

Now let's investigate the density parameter to \textbf{dbscan}. Let's
make it smaller and try \texttt{point\_eps\ =\ 0.2}.

Run the tests with:

\begin{itemize}
\tightlist
\item
  No PCA
\item
  dbscan clustering.
\item
  \textbf{Angles calculated before clustering.}
\item
  number of pathways 4.
\item
  Centroids not assigned in density based methods.
\end{itemize}

\includegraphics{test_clust_kmeans_files/figure-latex/unnamed-chunk-41-1.pdf}
\includegraphics{test_clust_kmeans_files/figure-latex/unnamed-chunk-41-2.pdf}
\includegraphics{test_clust_kmeans_files/figure-latex/unnamed-chunk-41-3.pdf}
\includegraphics{test_clust_kmeans_files/figure-latex/unnamed-chunk-41-4.pdf}

\hypertarget{test21}{%
\subsubsection{Test 21}\label{test21}}

Let's test the density parameter again but this time increase to
\texttt{point\_eps\ =\ 0.8}

Run the tests with:

\begin{itemize}
\tightlist
\item
  No PCA
\item
  dbscan clustering.
\item
  \textbf{Angles calculated before clustering.}
\item
  number of pathways 4.
\item
  Centroids not assigned in density based methods.
\end{itemize}

\includegraphics{test_clust_kmeans_files/figure-latex/unnamed-chunk-43-1.pdf}
\includegraphics{test_clust_kmeans_files/figure-latex/unnamed-chunk-43-2.pdf}
\includegraphics{test_clust_kmeans_files/figure-latex/unnamed-chunk-43-3.pdf}
\includegraphics{test_clust_kmeans_files/figure-latex/unnamed-chunk-43-4.pdf}

\end{document}
